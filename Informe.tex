\documentclass[]{article}

\usepackage{amsmath}
\usepackage{amssymb}
%opening
\title{}
\author{}

\begin{document}

\maketitle

\begin{abstract}

\end{abstract}

\section*{Ejercicio 2}
Rotación $R_x(\alpha)$:
\begin{equation}
	R_x(\alpha) = \begin{bmatrix}
		1 & 0 & 0 \\ 
		0 & cos(\alpha) & -\sin(\alpha) \\ 0 & \sin(\alpha) & \cos(\alpha)
	\end{bmatrix} = \begin{bmatrix}
		1 & 0 & 0 \\
		0 & -0,2225209340 & -0,9749279122 \\ 
		0 & 0,9749279122 & 0,2225209340
	\end{bmatrix}
\end{equation}

Rotación $R_x(\alpha)R_y(\beta)R_z(\gamma)$:
\begin{equation}
	\begin{bmatrix}
		3.0616e-17 &  5.3029e-17  & 1.0000e+00 \\
		6.8017e-01 &  7.3305e-01 & -5.9697e-17 \\
		-7.3305e-01 &  6.8017e-01 & -1.3625e-17
	\end{bmatrix}
\end{equation}

\textbf{Calculo yaw pitch y roll}

yaw $\gamma$:
\begin{equation}
	\gamma = \arctan_2(R_{12},R_{11}) = 0/0 \text{~Kaputt}
\end{equation}

pitch $\beta$:
\begin{equation}
	\beta = \arctan_2(-)
\end{equation}

\section*{Ejercicio 3}

Para pasar de A a W
\begin{equation}
	{}^W\xi_A = \begin{bmatrix}
		\cos(\theta) & -\sin(\theta) & A_x \\
		\sin(\theta) & \cos(\theta) & A_y \\ 0 & 0 & 1
	\end{bmatrix}
\end{equation}
Para pasar de W a A
\begin{equation}
	{}^A\xi_W = {}^W\xi_A^{-1}
\end{equation}
\end{document}
